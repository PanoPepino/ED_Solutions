
\section{Radiation and Relativistic Dynamics}

\subsection{\hyperref[Emission Rates by Lorentz Transformation]{Emission Rates by Lorentz Transformation}}

An electron enters and exits a capacitor with parallel-plate separation $d$ through two small holes. The electron velocity is given by $v \hat{z}$ and it is parallel to the capacitor electric field $\vec{E}$. The change in the electron velocity is small. Calculate the total energy $\Delta U'_{EM}$ and its linear momentum $\Delta P'_{EM}$ that was radiated by the electron in both rest and laboraty frames ($\Delta U_{EM}$ and  $\Delta P_{EM}$ respectively).

\subsection{\hyperref[A Merry Go Round of Radiating Particles]{A Merry Go Round of Radiating Particles}}

N identical, equally spaced\footnote{Is coronavirus still around?} point particles, each with a charge $q$, move in a circle of radius $a$. All of them have the same constant speed $v$ around the ring. Show that the Lienard-Wiechert electric field is \textit{static} everywhere on the symmetry axis.

\subsection{\hyperref[The Direction of the Velocity Field]{The Direction of the Velocity Field}}

Prove that the "velocity" part of the Lienard-Wiechert electric field points to the observer from the "anticipated position" of the moving point charge. The latter is the position the charge \textit{would} have moved \textit{if} it retained the velocity $\vec{v}_{ret}$ from $t = t_{ret}$ to the present time of observation.

\subsection{\hyperref[Radiating 14.4 Jackson Problem]{Radiating 14.4 Jackson Problem}}

Using the Liénard - Wiechert electric field, discuss the time-averaged power radiated per unit solid angle by a charged particle ($e^{-}$) in a \textbf{non-relativistic} motion in the next two different cases:

\begin{enumerate}
	\item Along the $z$ axis with position given by $z(t)= a cos (\omega t)$,
	\item In a circle of radius $R$ in the plane $xy$ with constant angular frecuency $w_{0}$.
\end{enumerate} 

\subsection{\hyperref[A Fast Particle in a Constant Electric Field]{A Fast Particle in a Constant Electric Field}}

A relativistic point particle with charge $q$ and mass $m$ moves in response to a uniform electric field $\mathbf{E}=E \hat{\mathbf{z}}$. The initial energy, linear momentum, and velocity are $\mathcal{E}_{0}, p_{0}$, and $\mathbf{u}(0)=u_{0} \hat{\mathbf{y}} .$ Find $\mathbf{r}(t)$ and show that eliminating $t$ gives the particle trajectory

\begin{equation}
z=\frac{\mathcal{E}_{0}}{q E} \cosh \left(\frac{q E y}{c p_{0}}\right).
\end{equation}

Check the non-relativistic limit.

\subsection{\hyperref[A Ringy Radiating Problem]{A Ringy Radiating Problem}}

\begin{enumerate}
	\item A small current loop moves with constant velocity $\mathbf{v}_{0}$ as viewed in the laboratory frame. Find the vector potential $\mathbf{A}(\mathbf{r})$ and the scalar potential $\varphi(\mathbf{r})$ in the lab frame. It may be convenient to introduce the vector $\mathbf{R}=\mathbf{r}-\mathbf{v}_{0} t$
	\item Take the limit $v_{0} \ll c$ in your formulae and deduce that the moving loop possesses both a magnetic dipole moment and an electric dipole moment.
\end{enumerate}






