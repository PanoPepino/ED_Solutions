\section{Electrostatics}

\subsection*{\hyperref[Conducting ball]{Conducting ball}}

A conducting ball of radius $R$ and total charge $Q$ sits in a homogeneous electric field $\vec{E}=E_{0} \hat{z}$. How does the electric field change by the presence of the ball? (Make an Ansatz of the form $\Phi(r, \theta, \phi)=f_{0}(r)+f_{1}(r) \cos \theta$ and motivate it.) Tip: $\hat{Z}=\cos \theta \hat{r}+\sin \theta \hat{\theta}$.

\subsection*{\hyperref[Conducting ball Again]{Conducting ball Again}}

 \begin{enumerate}
	\item A point charge $q$ sits at $\vec{a}$ inside a conducting uncharged sphere that is earthed with radius $R(|\vec{a}|<R)$.Compute the potential and the electric field inside the sphere using the method of mirror charges. Compute also the induced charge density on the surface of the sphere and show that the total charge on the surface is $-q$. What does the Gauss theorem say about the electrical field outside the sphere?
	\item Do the same analysis with the change that the sphere is isolated and uncharged. Tip: Determine the electric field outside the sphere with the new b.c.
	\item Follow again the same procedure as b for a sphere that is isolated and with charge $Q$.
\end{enumerate}

\subsection*{\hyperref[The Capacitance of an off-centered Capacitor]{The Capacitance of an off-centered Capacitor}} 

A spherical conducting shell centered at the origin has radius $R_{1}$ and is maintained at potential $V_{1}$. A second spherical conducting shell maintained at potential $V_{2}$ has radius $R_{2}>R_{1}$ but is centered at the point $s \:\hat{\mathbf{z}}$ where $s \ll R_{1}$.

\begin{enumerate}
	\item To lowest order in $s$, show that the charge density induced on the surface of the inner shell is

	\begin{equation}
		\sigma(\theta)=\epsilon_{0} \frac{R_{1} R_{2}\left(V_{2}-V_{1}\right)}{R_{2}-R_{1}}\left[\frac{1}{R_{1}^{2}}-\frac{3 s}{R_{2}^{3}-R_{1}^{3}} \cos \theta\right].
	\end{equation}

	Hint: Show first that the boundary of the outer shell is $r_{2} \approx R_{2}+s \cos \theta$.

	\item To lowest order in $s$, show that the force exerted on the inner shell is:
	
	\begin{equation}
		\mathbf{F}=\int d S \frac{\sigma^{2}}{2 \epsilon_{0}} \hat{\mathbf{n}}=\hat{\mathbf{z}} 2 \pi R_{1}^{2} \int^{\pi} d \theta \sin \theta \frac{\sigma^{2}(\theta)}{2 \epsilon_{0}} \cos \theta=-\frac{Q^{2}}{4 \pi \epsilon_{0}} \frac{s \hat{\mathbf{z}}}{R_{2}^{3}-R_{1}^{3}}.
	\end{equation}
\end{enumerate}

\subsection*{\hyperref[Spherical cavity and spherical functions]{Spherical cavity and spherical functions}}

Consider a sphere of radius $a$ where the surface of the upper hemisphere has a potential $+\Phi_{0}$ and the surface of the lower hemisphere has a potential $-\Phi_{0}$. In this case the Green Function is given by:

\begin{equation}
	G\left(r, r^{\prime}\right)=\frac{1}{\left|\vec{r}-\vec{r^{\prime}}\right|}-\frac{a}{r^{\prime}\left|\vec{r}-\frac{a^{2}}{r^{\prime 2}} \vec{r^{\prime}}\right|},
 \end{equation}

 where $\vec{r^{\prime}}$ refers to a unit source outside the sphere and $\vec{r}$ to the point where the potential is evaluated.

 \begin{enumerate}
	\item  Using the expression for the expansion of $\frac{1}{\left|\vec{r}-\vec{r^{\prime}}\right|}$ in the appropriate basis show that the Green's function can be written as
	
	\begin{equation}
		G\left(r, r^{\prime}\right)=4 \pi \sum_{l, m} \frac{1}{2 l+1}\left[\frac{r_{<}^{l}}{r_{>}^{l+1}}-\frac{1}{a}\left(\frac{a^{2}}{r r^{\prime}}\right)^{l+1}\right] Y_{l, m}^{*}\left(\theta^{\prime}, \phi^{\prime}\right) Y_{l, m}(\theta, \phi),
	\end{equation}

	\item Using Dirichlet boundary conditions, show that the potential outside the sphere has fol- lowing the expansion.
	
	\begin{equation}
		\Phi(r, \theta, \phi)=\sum_{l m}\frac{l+1}{a^{2} (2l+1)}\left(\frac{a}{r}\right)^{l+1} Y_{l, m}(\theta, \phi) \int \Phi_{0}\left(\theta^{\prime}, \phi^{\prime}\right) Y_{l, m}^{*}\left(\theta^{\prime}, \phi^{\prime}\right) d \Sigma^{\prime},
	\end{equation}

	which tends to 0 as $r \rightarrow \infty$.
\end{enumerate}

\subsection*{\hyperref[Green's function between concentric spheres]{Green's function between concentric spheres}}

Consider the green's function for Newnmann b.c. in the volume $\mathrm{V}$ between two concentric spheres between $r=a$ and $r=b, a<b$. We write the potential as

\begin{equation}
	\Phi(x)=\frac{1}{4 \pi \epsilon_{0}} \int_{V} \rho\left(x^{\prime}\right) G\left(x, x^{\prime}\right) d^{3} x^{\prime}+\frac{1}{4 \pi} \oint_{S} \frac{\partial \Phi}{\partial n^{\prime}} G d a^{\prime},
\end{equation}

where $S$ is the surface of the boundary. This implies that the b.c. for the Green's function is given by:

\begin{equation}
	\frac{\partial}{\partial n^{\prime}} G\left(x, x^{\prime}\right)=-\frac{4 \pi}{S},
\end{equation}

or $x^{\prime}$ in $S$. Expanding the Green's function in spherical harmonics we get:

\begin{equation}
	G\left(x, x^{\prime}\right)=\sum_{l=0}^{\infty} g_{l}\left(r, r^{\prime}\right) P_{l}(\cos \gamma),
\end{equation}

where $g_{l}\left(r, r^{\prime}\right)=\frac{r_{l}^{l}}{r_{>}^{l+1}}+f_{l}\left(r, r^{\prime}\right)$, and $\gamma$ is the angle between the vector $x$ and $x^{\prime}$.

Also here one can prove that $P_{l}(\cos \gamma)=\frac{4 \pi}{2 l+1} \sum_{m} Y_{l, m}^{*}\left(\theta^{\prime}, \phi^{\prime}\right) Y_{l, m}(\theta, \phi)$.

\begin{enumerate}
	\item Show for $l>0$ that the Green's function takes the symmetric form:
	
	\begin{equation}
		g_{l}\left(r, r^{\prime}\right)=\frac{r_{<}^{l}}{r_{>}^{l+1}}+\frac{1}{b^{2 l+1}-a^{2 l+1}}\left[\frac{l+1}{l}\left(r r^{\prime}\right)^{l}+\frac{l}{l+1} \frac{(a b)^{2 l+1}}{\left(r r^{\prime}\right)^{l+1}}+a^{2 l+1}\left(\frac{r^{l}}{r^{\prime l+1}}+\frac{r^{\prime l}}{r^{l+1}}\right)\right]
	\end{equation}

	\item Use the Green's function that you found in the situation that you have a normal electric field $E_{r}=-E_{0} \cos \theta$ at $r=b$ and $E_{r}=0$ at $r=a$. Show that the potential inside $V$ is
	
	\begin{equation}
		\Phi(x)=E_{0} \frac{r \cos \theta}{1-p^{3}}\left(1+\frac{a^{3}}{2 r^{3}}\right),
	\end{equation}

	where $p=\frac{a}{b}$. Find also for the electric field that:
	
	\begin{equation}
		E_{r}(r, \theta)=-E_{0} \frac{\cos \theta}{1-p^{3}}\left(1+\frac{a^{3}}{r^{3}}\right), \quad E_{\theta}(r, \theta)=E_{0} \frac{\sin \theta}{1-p^{3}}\left(1+\frac{a^{3}}{2 r^{3}}\right) .
	\end{equation}
\end{enumerate}